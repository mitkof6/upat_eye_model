\documentclass[11pt,a4paper,draft=false]{report}
\usepackage{amsmath,amsthm,amssymb}
\usepackage{bm,txfonts}       % bm: bold for Greek letters, txfonts -> varmathbb
\usepackage{breqn}              % automatically brake line in equations
\usepackage{cancel}
\usepackage{arydshln}         % matrix dash lines
\usepackage{booktabs}         % tables
\usepackage{color,soul}         % used for \hl{}
\usepackage[inline]{enumitem}   % use to set labels in the enumerate environment
\usepackage{fullpage}
\usepackage[pdftex]{graphicx}
\graphicspath{{./figures/}{./tikz/}}
\DeclareGraphicsExtensions{.pdf,.png,.jpg}
\usepackage{caption}
\captionsetup{labelfont=bf}
\usepackage[caption=false,labelfont=bf]{subfig}
\usepackage{listings}
% \usepackage{algorithm}
% \usepackage{algorithmic}
\usepackage{url}
\usepackage[pdfpagelabels,pdfusetitle,colorlinks=true,pdfborder={0 0
  0}]{hyperref}
\usepackage{natbib}
\bibliographystyle{IEEEtranN}

% indexing
%%%%%%%%%%%%%%%%%%%%%%%%%%%%%%%%%%%%%%%%%%%%%%%%%%%%%%%%%%%%%%%%%%%%%%%%%%%%%%%%

% \usepackage{makeidx}
% \makeindex

% make use of \index{word} in text

% acronyms
%%%%%%%%%%%%%%%%%%%%%%%%%%%%%%%%%%%%%%%%%%%%%%%%%%%%%%%%%%%%%%%%%%%%%%%%%%%%%%%%

\usepackage[acronym]{glossaries}

\newacronym{tscmc}{TSCMC}{Task Space Computed Muscle Control}
\newacronym{cmc}{CMC}{Computed Muscle Control}
\newacronym{dofs}{DoFs}{Degrees of Freedom}
\newacronym{dof}{DoF}{Degree of Freedom}
\newacronym{ik}{IK}{Inverse Kinematics}
\newacronym{tsik}{TSIK}{Task Space Inverse Kinematics}
\newacronym{rms}{RMS}{Root Mean Square}
\newacronym{fd}{FD}{Forward Dynamics}
\newacronym{id}{ID}{Inverse Dynamics}
\newacronym{md}{MD}{Mixed Dynamics}
\newacronym{jra}{JRA}{Joint Reaction Analysis}
\newacronym{eoms}{EoMs}{Equations of Motion}
\newacronym{emg}{EMG}{Electromyography}
\newacronym{mocap}{MoCap}{Motion Capture}
\newacronym{imu}{IMU}{Inertial Measurement Unit}
\newacronym{itsc}{ITSC}{Inverse Task Space Controller}
\newacronym{so}{SO}{Static Optimization}
\newacronym{pd}{PD}{Proportional Derivative}
\newacronym{dmc}{DMC}{Direct Marker Control}
\newacronym{dae}{DAE}{Differential Algebraic Equations}
\newacronym{rra}{RRA}{Residual Reduction Algorithm}
\newacronym{ode}{ODE}{Ordinary Differential Equations}
\newacronym{svd}{SVD}{Singular Value Decomposition}
\newacronym{mpp}{MPP}{Moore-Penrose Pseudoinverse}
\newacronym{cns}{CNS}{Central Nervous System}
\newacronym{umh}{UMH}{Uncontrolled Manifold Hypothesis}
\newacronym{ke}{KE}{Kinetic Energy}
\newacronym{pe}{PE}{Potential Energy}
\newacronym{rmse}{RMSE}{Root Mean Square Error}
\newacronym{em}{EOMs}{Extraocular Muscles}
\makeglossaries{}

% custom commands
%%%%%%%%%%%%%%%%%%%%%%%%%%%%%%%%%%%%%%%%%%%%%%%%%%%%%%%%%%%%%%%%%%%%%%%%%%%%%%%%

\DeclareMathAlphabet\mathbfcal{OMS}{cmsy}{b}{n} % mathcal bf
\newcommand{\ten}[1]{\mathbfcal{#1}}
\newcommand{\mat}[1]{\bm{#1}}
\renewcommand*{\vec}[1]{\bm{#1}}
\newcommand{\pd}[2]{\frac{\partial#1}{\partial#2}}
\newcommand{\subspace}[1]{\varmathbb{#1}}
\newcommand{\rs}[1]{\subspace{R} (#1)}
\newcommand{\cs}[1]{\subspace{C} (#1)}
\newcommand{\ns}[1]{\subspace{N} (#1)}
\newcommand{\fs}[1]{\subspace{F}}
\newcommand{\ms}[1]{\subspace{M}}
\newcommand{\R}[1]{\mathfrak{R}^{#1}}
\newcommand{\inr}[1]{\in\R{#1}}
\newcommand{\TP}[1]{\mat{T}_{#1}}     % tall projection
\newcommand{\FP}[1]{\mat{F}_{#1}}     % fat projection
\newcommand{\NT}[1]{\mat{N}_{\TP{#1}}}
\newcommand{\N}[1]{\mat{N}_{#1}}
\newcommand{\dNT}[1]{\mat{\dot{N}}_{\TP{#1}}}
\newcommand{\NF}[1]{\mat{N}_{\FP{#1}}}
\newcommand{\dNF}[1]{\mat{\dot{N}}_{\FP{#1}}}
\newcommand{\mc}[1]{\mathcal{#1}}

% mathematical proofs
\newtheorem{theorem}{Theorem}[chapter]
\newtheorem{corollary}{Corollary}[chapter]
\newcommand{\corollaryautorefname}{Corollary}
\newtheorem{proposition}{Proposition}[chapter]
\newcommand{\propositionautorefname}{Proposition}
\newtheorem{remark}{Remark}[chapter]
\newcommand{\remarkautorefname}{Remark}
\newtheorem{definition}{Definition}[chapter]
\newcommand{\definitionautorefname}{Definition}
\newcommand{\subfigureautorefname}{Figure} % subfloat
\renewcommand*{\figureautorefname}{Figure}
\renewcommand*{\sectionautorefname}{Section}
\renewcommand*{\subsectionautorefname}{Subsection}
\newcommand{\algorithmautorefname}{Algorithm}
\makeatletter
% \renewcommand\thealgorithm{\thechapter.\arabic{algorithm}}
\@addtoreset{algorithm}{chapter}
\makeatother
\def\equationautorefname~#1\null{Equation #1\null}

% \renewcommand{\baselinestretch}{1.5} % line stretch

%%%%%%%%%%%%%%%%%%%%%%%%%%%%%%%%%%%%%%%%%%%%%%%%%%%%%%%%%%%%%%%%%%%%%%%%%%%%%%%%
\title{An Open-Source OpenSim Oculomotor Model for Kinematic and Dynamic
  Simulation}

\author{Constantinos Filip, Dimitar Stanev\footnote{Electrical and Computer
    Engineering Department, University of Patras, Greece, Corresponding author:
    stanev@ece.upatras.gr} and Konstantinos Moustakas}

\date{\today}

\begin{document}
%%%%%%%%%%%%%%%%%%%%%%%%%%%%%%%%%%%%%%%%%%%%%%%%%%%%%%%%%%%%%%%%%%%%%%%%%%%%%%%%

\maketitle

%%%%%%%%%%%%%%%%%%%%%%%%%%%%%%%%%%%%%%%%%%%%%%%%%%%%%%%%%%%%%%%%%%%%%%%%%%%%%%%%
\begin{abstract}
  bla
\end{abstract}

%%%%%%%%%%%%%%%%%%%%%%%%%%%%%%%%%%%%%%%%%%%%%%%%%%%%%%%%%%%%%%%%%%%%%%%%%%%%%%%%
\section*{Introduction}\label{sec:introduction}

%%%%%%%%%%%%%%%%%%%%%%%%%%%%%%%%%%%%%%%%%%%%%%%%%%%%%%%%%%%%%%%%%%%%%%%%%%%%%%%%
\section*{Methods}\label{sec:methods}

%%%%%%%%%%%%%%%%%%%%%%%%%%%%%%%%%%%%%%%%%%%%%%%%%%%%%%%%%%%%%%%%%%%%%%%%%%%%%%%%
\subsection*{Eye Modeling}\label{sec:eye-Modeling}

he orbital plant consists of the globe (eyeball), three pairs of extraocular
muscles, and connective tissues. The size of an emmetropic human adult eye is
approximately 24.2 mm (transverse, horizontal) x 23.7 mm (sagittal, vertical) x
22.0–24.8 mm (axial, anteroposterior) with no significant difference between
sexes and age groups. In the transverse diameter, the eyeball size may vary from
21 mm to 27 mm. Thus, it can be approximated by a solid sphere with 12mm
radius. The eyeball was constructed in Blender, an open-source software 3D
creation Software. We used a spherical mesh with 32 segments and 12 rings, to
construct the vitreous humor (body) as solid sphere and a conical plate to
construct the cornea. The weight of an average human eye is 7.5grams and the
moment of inertia can be calculated similarly as in the case of a spherical
homogenous and isotropic object with radius 12mm ($\mat{I} = 2/5 m r^2$ at the
center of mass).

%%%%%%%%%%%%%%%%%%%%%%%%%%%%%%%%%%%%%%%%%%%%%%%%%%%%%%%%%%%%%%%%%%%%%%%%%%%%%%%%
\subsection*{Muscle Modeling}\label{sec:muscle-modeling}

\begin{table}[h]
  \caption{TODO}\label{tab:muscle-path}
  \begin{tabular}{@{}cccccccccc@{}}
    \toprule
    \textbf{Muscle} & \multicolumn{3}{c}{\textbf{Origin}}                                & \multicolumn{3}{c}{\textbf{Pulley}}                                & \multicolumn{3}{c}{\textbf{Insertion}}                                                                      \\ \midrule
                    & \textit{\textbf{Ox}} & \textit{\textbf{Oy}} & \textit{\textbf{Oz}} & \textit{\textbf{Px}} & \textit{\textbf{Py}} & \textit{\textbf{Pz}} & \textit{\textbf{Ix}} & \multicolumn{1}{l}{\textit{\textbf{Iy}}} & \multicolumn{1}{l}{\textit{\textbf{Iz}}} \\ \midrule
    LR              & -0.034               & 0.0006               & -0.013               & -0.0102              & 0.0003               & 0.012                & 0.0065               & \multicolumn{1}{l}{0}                    & \multicolumn{1}{l}{0.0101}               \\
    MR              & -0.030               & 0.0006               & -0.017               & -0.0053              & 0.00014              & -0.0146              & 0.0088               & \multicolumn{1}{l}{0}                    & \multicolumn{1}{l}{-0.0096}              \\
    SR              & -0.0317              & 0.0036               & -0.016               & -0.0092              & 0.012                & -0.002               & 0.0076               & 0.0104                                   & 0                                        \\
    IR              & -0.0317              & -0.0024              & -0.016               & -0.0042              & -0.0128              & -0.0042              & 0.00805              & -0.0102                                  & 0                                        \\
    SO              & 0.0082               & 0.0122               & -0.0152              & -0.030834            & 0.001145             & -0.01644             & 0.0044               & 0.011                                    & 0.0029                                   \\
    IS              & 0.0113               & -0.0154              & -0.0111
                    & -0.00718             & -0.0135              & 0
                                                                                         &
                                                                                           -0.008
                                                                                                                & 0                                        & 0.009 \\
    \bottomrule
  \end{tabular}
\end{table}


%%%%%%%%%%%%%%%%%%%%%%%%%%%%%%%%%%%%%%%%%%%%%%%%%%%%%%%%%%%%%%%%%%%%%%%%%%%%%%%%
\subsection*{Passive Connective Tissues}\label{sec:passive-connective-tissues}

The passive connective tissues of the eyeball apply a restoring force, which
brings the globe back to the central position when the net force from the
\gls{em} is zero. These tissues include all non-muscular suspensory tissues,
such as Tenon’s capsule, the optic nerve, the fat pad and the conjunctiva. The
force-displacement curve of the net elasticity can be represented as

\begin{equation}\label{equ:passive-tissue}
  \vec{f}_t = -k_p \vec{q} - k_c 10^{-3} \vec{q}^3 - k_d * \vec{\dot{q}}
\end{equation}
%
where, $\vec{f}_t$ represents the passive tissue forces,
$k_p= 0.33 g / deg$ , $k_p= 1.56 g / (deg^3)$  (\hl{TODO: convert to
	$Nm / rad$}) and
$k_v= 10^{-4} Nm s / rad$ the constants and $\vec{\dot{q}} \inr{3}$ the
rotational coordinates of the model. These forces serve the eye’s stabilization,
and are modeled using \texttt{OpenSim}'s expression based coordinate force.
% 0.33 * 9.8066500286389 * 10**-3 / (numpy.pi / 180) = 0.18542028707494282
% 1.56 * 9.8066500286389 * 10**-3 / (numpy.pi / 180) ** 3 = 2877.4856893811248

%%%%%%%%%%%%%%%%%%%%%%%%%%%%%%%%%%%%%%%%%%%%%%%%%%%%%%%%%%%%%%%%%%%%%%%%%%%%%%%%
\section*{Results}\label{sec:results}

%%%%%%%%%%%%%%%%%%%%%%%%%%%%%%%%%%%%%%%%%%%%%%%%%%%%%%%%%%%%%%%%%%%%%%%%%%%%%%%%
\subsection*{Model Validation}\label{sec:model-validation}

To meet good fidelity criteria, a model requires to be verified and
validated. In our study, verification in oculomotor models was performed by
comparing the force-length characteristic curves of the modeled \gls{em} to
published data. These data include only the lateral rectus characteristic
curves. However, we can make a safe assumption that the other muscles have
similar properties. More characteristic curves showing the force-length
relationship of the other \gls{em}, as well as the changes in muscle length with
rotation in all directions, are presented in the end of this report. Further
verification can be done by comparing the model’s joint forces produced during
horizontal movement with clinically collected data, but these data are not
available.  Validation was performed by simulating and analyzing synthesized eye
movements while ensuring that Listing Law was obeyed by maintaining zero torsion
in secondary gaze positions, and by testing if the assessed innervations are
sufficient in fixating the eye at a desired position.

%%%%%%%%%%%%%%%%%%%%%%%%%%%%%%%%%%%%%%%%%%%%%%%%%%%%%%%%%%%%%%%%%%%%%%%%%%%%%%%%
\subsection*{Fixation Controller}\label{sec:fixation-controller}

%%%%%%%%%%%%%%%%%%%%%%%%%%%%%%%%%%%%%%%%%%%%%%%%%%%%%%%%%%%%%%%%%%%%%%%%%%%%%%%%
% \section*{Discussion}\label{sec:discussion}

%%%%%%%%%%%%%%%%%%%%%%%%%%%%%%%%%%%%%%%%%%%%%%%%%%%%%%%%%%%%%%%%%%%%%%%%%%%%%%%%
\section*{Conclusion}\label{sec:conclusion}

We manage to construct a complete ocular model that represents the ocular
motility of a normal human eye. The model can be used to drive simulations of
different eye movement systems and be derive some inference about the excitation
and activation patterns. The verification and validation of the model showed
that it matches the data found in literature and it is able to produce
synthetized movements. We even showed that the simulation results produced by
static optimization and forward dynamics in OpenSim are less satisfactory than
the proposed solution where we used a PD controller to minimize the tracking
error between the desired and estimated trajectory of saccadic movements. The
desired kinematics were tracked within a maximum \gls{rmse} of and in horizontal
and vertical saccades respectively. With the application of the controller, a
very rapid instantaneous acceleration and a constant velocity to sustain clear
vision is attained. The produced activation levels were in accordance with the
descriptions found in the bibliography and the highest activation levels were
shown only during the time intervals when the main agonist muscle was activated.

%%%%%%%%%%%%%%%%%%%%%%%%%%%%%%%%%%%%%%%%%%%%%%%%%%%%%%%%%%%%%%%%%%%%%%%%%%%%%%%%
% \bibliography{mylibrary}
% \addcontentsline{toc}{chapter}{Bibliography}


%%%%%%%%%%%%%%%%%%%%%%%%%%%%%%%%%%%%%%%%%%%%%%%%%%%%%%%%%%%%%%%%%%%%%%%%%%%%%%%%
\end{document}

%%% Local Variables:
%%% mode: latex
%%% TeX-master: t
%%% End:

\documentclass[11pt,a4paper,draft=false]{report}
\usepackage{amsmath,amsthm,amssymb}
\usepackage{bm,txfonts}       % bm: bold for Greek letters, txfonts -> varmathbb
\usepackage{breqn}              % automatically brake line in equations
\usepackage{cancel}
\usepackage{arydshln}         % matrix dash lines
\usepackage{booktabs}         % tables
\usepackage{color,soul}         % used for \hl{}
\usepackage[inline]{enumitem}   % use to set labels in the enumerate environment
\usepackage{fullpage}
\usepackage[pdftex]{graphicx}
\graphicspath{{./figures/}{./tikz/}}
\DeclareGraphicsExtensions{.pdf,.png,.jpg}
\usepackage{caption}
\captionsetup{labelfont=bf}
\usepackage[caption=false,labelfont=bf]{subfig}
\usepackage{listings}
% \usepackage{algorithm}
% \usepackage{algorithmic}
\usepackage{url}
\usepackage[pdfpagelabels,pdfusetitle,colorlinks=true,pdfborder={0 0
  0},pdfauthor={Dimitar Stanev}]{hyperref}
\usepackage{natbib}
\bibliographystyle{IEEEtranN}

% indexing
%%%%%%%%%%%%%%%%%%%%%%%%%%%%%%%%%%%%%%%%%%%%%%%%%%%%%%%%%%%%%%%%%%%%%%%%%%%%%%%% 

% \usepackage{makeidx}
% \makeindex

% make use of \index{word} in text

% acronyms
%%%%%%%%%%%%%%%%%%%%%%%%%%%%%%%%%%%%%%%%%%%%%%%%%%%%%%%%%%%%%%%%%%%%%%%%%%%%%%%% 

\usepackage[acronym]{glossaries}

\newacronym{dofs}{DoFs}{Degrees of Freedom}
\newacronym{dof}{DoF}{Degree of Freedom}
\newacronym{eoms}{EoMs}{Equations of Motion}
\newacronym{fd}{FD}{Forward Dynamics}
\newacronym{pd}{PD}{Proportional Derivative}
\newacronym{cns}{CNS}{Central Nervous System}
\newacronym{lr}{LR}{Lateral Rectus}
\newacronym{sr}{SR}{Superior Rectus}
\newacronym{ir}{IR}{Inferior Rectus}
\newacronym{mr}{MR}{Medial Rectus}
\newacronym{so}{SO}{Superior Oblique}
\newacronym{io}{IO}{Inferior Oblique}
\newacronym{em}{EOMs}{Extraocular Muscles}
\newacronym{fl}{F-L}{Force-Length}
\newacronym{fv}{F-V}{Force-Velocity}
\makeglossaries{}

% custom commands
%%%%%%%%%%%%%%%%%%%%%%%%%%%%%%%%%%%%%%%%%%%%%%%%%%%%%%%%%%%%%%%%%%%%%%%%%%%%%%%% 

\DeclareMathAlphabet\mathbfcal{OMS}{cmsy}{b}{n} % mathcal bf
\newcommand{\ten}[1]{\mathbfcal{#1}}
\newcommand{\mat}[1]{\bm{#1}}
\renewcommand*{\vec}[1]{\bm{#1}}
\newcommand{\R}[1]{\mathfrak{R}^{#1}}
\newcommand{\inr}[1]{\in\R{#1}}

\newcommand{\subfigureautorefname}{Figure} % subfloat
\renewcommand*{\figureautorefname}{Figure}
\renewcommand*{\sectionautorefname}{Section}
\renewcommand*{\subsectionautorefname}{Subsection}
\newcommand{\algorithmautorefname}{Algorithm}
\makeatletter
% \renewcommand\thealgorithm{\thechapter.\arabic{algorithm}}
\@addtoreset{algorithm}{chapter}
\makeatother
\def\equationautorefname~#1\null{Equation #1\null}

% in order to add newline in table cells
\usepackage{makecell}
\renewcommand\theadalign{bc}
\renewcommand\theadfont{\bfseries}
\renewcommand\theadgape{\Gape[4pt]}
\renewcommand\cellgape{\Gape[4pt]}

%%%%%%%%%%%%%%%%%%%%%%%%%%%%%%%%%%%%%%%%%%%%%%%%%%%%%%%%%%%%%%%%%%%%%%%%%%%%%%%% 
\title{An Open-Source OpenSim Oculomotor Model for Kinematics and Dynamics
  Simulation}

\author{Constantinos Filip, Dimitar Stanev\footnote{Electrical and Computer
    Engineering Department, University of Patras, Greece; Corresponding
    author: \url{stanev@ece.upatras.gr}}, and Konstantinos Moustakas}

\date{\today}

\begin{document}
%%%%%%%%%%%%%%%%%%%%%%%%%%%%%%%%%%%%%%%%%%%%%%%%%%%%%%%%%%%%%%%%%%%%%%%%%%%%%%%% 

\maketitle

%%%%%%%%%%%%%%%%%%%%%%%%%%%%%%%%%%%%%%%%%%%%%%%%%%%%%%%%%%%%%%%%%%%%%%%%%%%%%%%% 
\begin{abstract}
  Studying human eye movement has significant implications for improving our
  understanding of the oculomotor system and treating various visuomotor
  disorders. An open-source musculoskeletal model of the human eye, that can be
  used for kinematics and dynamics analysis, is implemented based on the data
  reported in literature and made publicly available\footnote{SimTK project:
    \url{https://simtk.org/projects/eye}}. The model is implemented in
  \texttt{OpenSim}~\cite{Delp2007}, which is an open-source framework for
  modeling and simulation of musculoskeletal systems. The calibration of the
  model parameters is based on physiological measurements of the human
  eye~\cite{Iskander2018}. The model incorporates an eye globe, orbital
  suspension tissues and six \gls{em}. The excitation and activation patterns
  for a variety of targets can be calculated using a closed-loop fixation
  controller that drives the model to perform saccadic movements in a \gls{fd}
  manner. The controller minimizes the error between the desired saccadic
  trajectory and the predicted movement. Consequently, this model enables the
  investigation muscle activation patterns during static fixation and analyze
  the dynamics of various eye movements.
\end{abstract}

%%%%%%%%%%%%%%%%%%%%%%%%%%%%%%%%%%%%%%%%%%%%%%%%%%%%%%%%%%%%%%%%%%%%%%%%%%%%%%%% 
\section*{Introduction}\label{sec:introduction}

Rapid and accurate eye movements are crucial for coordinated direction of
gaze~\cite{Lee2006}. Studying human eye movement has significant implications
for improving our understanding of the oculomotor system and treating visuomotor
disorders. Over the past decades, biomechanics simulation has provided the means
to analyze different human movements. The same principles can be used to analyze
visual tasks by modeling the musculoskeletal properties of the oculomotor
system. Consequently, this model can be used to investigate muscle activation
patterns during static fixation, analyze the dynamics of various eye movements,
calculate metabolic costs and simulate different eye disorders, such as
different forms of strabismus. Furthermore, it can be easily integrated with
available full body models in order to analyze the relation between the
vestibular and oculomotor systems.

Eye movements are a generated from a coordination of the six \gls{em}. Clinical
trials have provided a profound knowledge of how the \gls{em} act on the eye
globe~\cite{Robinson1969a}, the resistive tension of the surrounding tissues and
the length-tension relationship of the muscles~\cite{Collins1981}. Various
computational models of the extraocular muscles and orbital mechanics have been
proposed, which provide insight and scientific bases for oculomotor
biomechanics, control of eye movement and binocular misalignment. These models
focus on the realism of muscle behavior and they were based on the viscoelastic
properties and physiological data \gls{em}.

The first 3D biomechanical model was developed by~\cite{Robinson1964a,
  Robinson1969}, who simplified the formulation by considering the elasticity of
the \gls{em} ignoring their dynamics. The model incorporates anatomically
realistic muscle paths and empirical innervation-length-tension
relationships. To study the neural control of rapid saccadic movements, models
using anatomical and mechanical properties of \gls{em} have been developed by
accounting for the nonlinear muscle dynamics~\cite{Thelen2003a,
  Millard2013}. Such models, having the advantage of supporting dynamics
simulation, are used in conjunction with brain level
controllers~\cite{James2018}.

%%%%%%%%%%%%%%%%%%%%%%%%%%%%%%%%%%%%%%%%%%%%%%%%%%%%%%%%%%%%%%%%%%%%%%%%%%%%%%%% 
\section*{Methods}\label{sec:methods}

%%%%%%%%%%%%%%%%%%%%%%%%%%%%%%%%%%%%%%%%%%%%%%%%%%%%%%%%%%%%%%%%%%%%%%%%%%%%%%%% 
\subsection*{Eye Modeling}\label{sec:eye-Modeling}

The orbital plant consists of the globe (eyeball), three pairs of extraocular
muscles and the connective passive tissues. The size of an emmetropic human
adult eye is approximately 0.0242 m (transverse, horizontal), 0.0237 m
(sagittal, vertical), 0.022–0.0248 m (axial, anteroposterior) with no
significant difference between sexes and age groups. In the transverse diameter,
the eyeball size may vary from 0.021 m to 0.027 m. Thus, it can be approximated
by a solid sphere with 0.012 m radius. The eyeball was constructed in
\texttt{Blender}, an open-source software 3D creation software. We used a
spherical mesh with 32 segments and 12 rings, to construct the vitreous humor
(body) as solid sphere and a conical plate to construct the cornea. The weight
of an average human eye is 0.0075 kg and the moment of inertia can be calculated
similarly as in the case of a spherical homogeneous and isotropic object with
radius 0.012 m ($\mat{I} = 2/5 m r^2$ at the center of mass).

%%%%%%%%%%%%%%%%%%%%%%%%%%%%%%%%%%%%%%%%%%%%%%%%%%%%%%%%%%%%%%%%%%%%%%%%%%%%%%%% 
\subsection*{Muscle Modeling}\label{sec:muscle-modeling}

The six \gls{em}, including four rectus muscles and two oblique muscles, are
controlled by the cranial nerves so as to track a visual target and to stabilize
the image of the object. The \gls{lr} and \gls{mr} muscles form an antagonistic
pair to produce horizontal eye movements. The \gls{sr} and \gls{ir} muscles form
the vertical antagonist pair, which mainly controls vertical eye movement and
also affects rotation about the horizontal plane and the line of sight
(secondary action) due to insertion positions and the path of the muscles. The
\gls{so} muscle passes through the cartilaginous trochlea attached to the
orbital wall, which reflects the \gls{so} path by 51 deg. The \gls{io} muscle
originates from the orbital wall anteroinferior to the globe center and inserts
on the sclera posterior to the globe equator. The primary actions of \gls{so}
and \gls{io} cause rotation of the globe around the visual axis and vertical
movement.

The model relies on the passive pulley assumption in order to keep it simple and
provide faster simulation speed. \autoref{tab:muscle-path} shows the positions
of muscle pulleys, as well as the origin and insertion points of the \gls{em},
defined in the local body coordinates of the globe. The data is based on
physiological measurements~\cite{Iskander2018}, with some minor adaptation so as
to prevent penetration into the eye globe. Since no position was documented for
the origin of the \gls{so}, a point close to the origins of the rectus muscles
was chosen to match the fiber length in the primary position of the \gls{so}
muscle.

\begin{table}[ht]
  \caption{Muscle path for the six \gls{em} (dimensions are given in
    meters).}\label{tab:muscle-path}
  \begin{tabular}{@{}cccccccccc@{}}
    \toprule
    \textbf{Muscle}
    & \multicolumn{3}{c}{\textbf{Origin}}
    & \multicolumn{3}{c}{\textbf{Pulley}}
    & \multicolumn{3}{c}{\textbf{Insertion}} \\
    \midrule
    & \textit{\textbf{Ox}} & \textit{\textbf{Oy}} & \textit{\textbf{Oz}}
    & \textit{\textbf{Px}} & \textit{\textbf{Py}} & \textit{\textbf{Pz}}
    & \textit{\textbf{Ix}} & \textit{\textbf{Iy}} & \textit{\textbf{Iz}} \\
    \midrule
    \gls{lr} & -0.034 & 0.0006 & -0.013 & -0.0102 & 0.0003 & 0.012 & 0.0065 & 0 & 0.0101 \\
    \gls{mr} & -0.030 & 0.0006 & -0.017 & -0.0053 & 0.00014 & -0.0146 & 0.0088 & 0 & -0.0096 \\
    \gls{sr} & -0.0317 & 0.0036 & -0.016 & -0.0092 & 0.012 & -0.002 & 0.0076 & 0.0104 & 0 \\
    \gls{ir} & -0.0317 & -0.0024 & -0.016 & -0.0042 & -0.0128 & -0.0042 & 0.00805 & -0.0102 & 0 \\
    \gls{so} & 0.0082 & 0.0122 & -0.0152 & -0.030834 & 0.001145 & -0.01644 & 0.0044 & 0.011 & 0.0029 \\
    \gls{io} & 0.0113 & -0.0154 & -0.0111 & -0.00718 & -0.0135 & 0 & -0.008 & 0 & 0.009 \\
    \bottomrule
  \end{tabular}
\end{table}

The Millard muscle model~\cite{Millard2013} has been adopted for the modeling of
the \gls{em}, which allows to manually fit the reported data for the \gls{fl}
curves. The muscles were modeled using the rigid tendon assumption that ignores
the elasticity of the tendon. This means that the series element of the muscle
model is not included (the tendon length $l^T$ is equal to the tendon slack
length $l_s^T$). This assumption is valid when the ratio of the tendon length to
the muscle length is less or equal to one, as the in the case of all
\gls{em}. \gls{em} are considered parallel-fibered muscles, so the pennation
angle is zero ($\alpha = 0$). Maximum isometric force $f_o^M$, optimal fiber
length $l_o^M$ and tendon length $l^T$ are presented
in~\autoref{tab:muscle-path}.

The active and passive \gls{fl} curves for the \gls{em} differ from that of a
skeletal muscle. As shown in \autoref{fig:millard-curves}, we can fine-tune
these curves so as to fit the experimental data available for the gls{lr}
muscle. The following values where used for the active \gls{fl} curve:

\begin{itemize}
\item min norm active fiber length: 0.55
\item transition norm fiver length: 0.7
\item max norm active fiver length: 1.8
\item shallow ascending slope: 2.4
\item minimum value: 0.0
\end{itemize}
%
and for the passive \gls{fl} curve accordingly:

\begin{itemize}
\item strain at zero force: -0.18
\item strain at one norm force: 0.4
\end{itemize}

\begin{figure}[ht]
  \subfloat[Active \gls{fl} curve]{\includegraphics[width=0.5\textwidth,
    keepaspectratio]{active-force-length-curve.png}\label{fig:active-force-length-curve}}
  \subfloat[Passive \gls{fl} curve]{\includegraphics[width=0.5\textwidth,
    keepaspectratio]{passive-force-length-curve.png}\label{fig:passive-force-length-curve}}
  \caption{The active and passive \gls{fl} curve definition of the Millard
    muscle model as implemented in \texttt{OpenSim}.}\label{fig:millard-curves}
\end{figure}

The parameters that describe the above relationships were chosen to fit the
curves reported in~\cite{Iskander2018}, by matching the \gls{fl} relationship at
maximum activation of the \gls{lr} muscle. This represents the first part of
testing the fidelity of the model. Due to lack of data describing the other
muscles’ \gls{fl} relationships, we used the parameters found describing the
normalized curves of active and passive \gls{fl} relationships of the gls{lr}
for the other \gls{em} as well.

\gls{em} have a higher fraction of fast twitch fibers and thus different
\gls{fv} behavior, due to different structures compared to skeletal
muscles. Despite that, the default Millard \gls{fv} curve was used for the six
\gls{em}, since the behavior of the selected muscle model depends mainly on the
maximum contraction velocity $v^{\text{max}}$. The maximum muscle contraction
velocity is tuned so as to match the peak velocity of saccadic eye movement 15.7
rad / s. Following this definition, the maximum muscle contraction velocity is
given in optimal fiber length per seconds and it is thus different for each
\gls{em}, as their optimal fiber length is different. Furthermore, because of
the different structure of neural control of the eyes, activation and
deactivation delays ($\tau_d = 5$ ms) are lower than in skeletal
muscles. Finally, two separate wrapping spheres for the rectus muscles and the
oblique muscles were created, to avoid abnormal changes on the \gls{fl} curve as
the eyeball rotates in the three directions.

\begin{table}[ht]
  \caption{Millard muscle parameters for the
    \gls{em}.}\label{tab:muscle-parameters}
  \begin{tabular}{@{}cccccccccc@{}}
    \toprule
    \thead{Muscle}
    & \thead{Maximum Isometric \\ Force (N)}
    & \thead{Optimal Fiber \\ Length (m)}
    & \thead{Tendon Slack \\ Length (m)}
    & \thead{Maximum Contraction \\ Velocity (m / s)} \\
    \midrule
    \gls{lr} & 1.4710 & 0.04898 & 0.0084 & 3.8483 \\
    \gls{mr} & 1.5740 & 0.04084 & 0.0038 & 4.6155 \\
    \gls{sr} & 1.1768 & 0.04487 & 0.0054 & 4.2009 \\
    \gls{ir} & 1.4269 & 0.04549 & 0.0048 & 4.1437 \\
    \gls{so} & 0.6031 & 0.03956 & 0.0265 & 4.7648 \\
    \gls{io} & 0.5590 & 0.04110 & 0.0015 & 3.5863 \\
    \bottomrule
  \end{tabular}
\end{table}

%%%%%%%%%%%%%%%%%%%%%%%%%%%%%%%%%%%%%%%%%%%%%%%%%%%%%%%%%%%%%%%%%%%%%%%%%%%%%%%% 
\subsection*{Passive Connective Tissues}\label{sec:passive-connective-tissues}

The passive connective tissues of the eyeball apply a restoring force, which
brings the globe back to the central position when the net force from the
\gls{em} is zero. These tissues include all non-muscular suspensory tissues,
such as Tenon’s capsule, the optic nerve, the fat pad and the conjunctiva. The
force-displacement curve of the net elasticity can be represented as

\begin{equation}\label{equ:passive-tissue}
  \vec{f}_t = -k_p \vec{q} - k_c 10^{-3} \vec{q}^3 - k_d * \vec{\dot{q}}
\end{equation}
% 
where, $\vec{f}_t$ represents the passive tissue forces, $k_p= 0.002225$ N m /
rad, $k_c= 34.5297$ N m / (rad\textsuperscript{3}) and $k_v= 0.002$ N m s / rad
the constants and $\vec{\dot{q}} \inr{3}$ the rotational coordinates of the
model~\cite{Collins1981}. These forces serve the eye’s stabilization, and are
modeled using \texttt{OpenSim}'s expression based coordinate force.

% 0.33 * 9.8066500286389 * 10**-3 / (numpy.pi / 180) = 0.18542028707494282 * 0.012
% 1.56 * 9.8066500286389 * 10**-3 / (numpy.pi / 180) ** 3 = 2877.4856893811248 * 0.012

%%%%%%%%%%%%%%%%%%%%%%%%%%%%%%%%%%%%%%%%%%%%%%%%%%%%%%%%%%%%%%%%%%%%%%%%%%%%%%%% 
\section*{Results}\label{sec:results}

%%%%%%%%%%%%%%%%%%%%%%%%%%%%%%%%%%%%%%%%%%%%%%%%%%%%%%%%%%%%%%%%%%%%%%%%%%%%%%%% 
\subsection*{Fixation Controller}\label{sec:fixation-controller}

A fixation controller was implemented as a custom \texttt{OpenSim} plugin. The
parameters of the controller are: the desired horizontal $\theta_H$ and vertical
$\theta_V$ fixation angles (in degrees), the saccade onset and velocity, and the
gains of \gls{pd} tracking controller ($k_p$, $k_d $). A sigmoid function is
used for generating smooth saccade trajectories

\begin{equation}\label{equ:sigmoid}
  \begin{aligned}
    \theta_d(t) &= \frac{a}{2} \Big(\tanh(b (t - t_0)) + 1\Big) \\
    \dot{\theta}_d(t) &= \frac{a b}{2} \Big(\tanh^2(b (t - t_0)) - 1\Big) 
  \end{aligned}
\end{equation}
%
where $\theta_d(t)$ and $\dot{\theta}_d(t)$ represent the desired orientation
and velocity at time $t$, $a$ the magnitude of the trajectory, $b$ the slope and
$t_0$ a time shift constant. Provided a fixation goal $\theta_g$ and a desired
saccade velocity $\dot{\theta}_g$ the parameters of the sigmoid function are
defined as follows $a = \theta_g$, $b = 2 \dot{\theta}_g / \theta_g$. The
\gls{pd} tracking controller has the following form

\begin{equation}\label{equ:pd-controller}
  \ddot{\theta}(t) = k_p (\theta_d(t) - \theta(t)) + k_d (\dot{\theta}_d(t) -
  \dot{\theta}(t)).
\end{equation}

The sign and magnitude of $\ddot{\theta}(t)$ both for the horizontal and
vertical coordinates of the fixation target is used to actuate the corresponding
muscles in order to achieve the desired goal. \autoref{fig:model} presents an
instance of the model during simulation with the corresponding muscles
activated. \autoref{fig:simulated-saccade} depicts the simulated coordinates,
speeds and estimated \gls{em} activations that reproduce the desired saccade
trajectory.

\begin{figure}[ht]
  \centering
  \includegraphics[width=.6\textwidth]{model.png}
  \caption{Model with a fixation at $\theta_H = -15$ deg $\theta_V = 15$ deg
    during simulation. Blue denotes low and red high activation
    values.}\label{fig:model}
\end{figure}

\begin{figure}[ht]
  \centering
  \includegraphics[width=1.\textwidth]{UPAT_Eye_Model_Passive_Pulleys_v3_States[-15][15].pdf}
  \caption{Simulated saccade response with a fixation at $\theta_H = -15$ deg
    $\theta_V = 15$ deg. The left subplot represents the simulated generalized
    coordinates; the middle the coordinate speeds; right the estimated \gls{em}
    activations.}\label{fig:simulated-saccade}
\end{figure}

%%%%%%%%%%%%%%%%%%%%%%%%%%%%%%%%%%%%%%%%%%%%%%%%%%%%%%%%%%%%%%%%%%%%%%%%%%%%%%%% 
\section*{Conclusion}\label{sec:concluison}

A realistic oculomotor model representing the motility of a normal human eye was
presented and made publicly available. The parameters of the model were
calibrated using available experimental measured data. The model can be used for
kinematics and dynamics analysis or as a tool for obtaining the muscle
activations that generate a desired saccade, using a closed-loop fixation
controller in a \gls{fd} manner. There is of course space for further
improvement, which will enhance the accuracy and the predictability of the
proposed computational eye model. In this study, we didn't attempt to model the
muscle pulleys~\cite{Kono2002a}, which vary as a function of the model
coordinates. Furthermore, the users should consider validation of the eye model
based on the requirements of the targeted utility of the model and the variables
of interests.

%%%%%%%%%%%%%%%%%%%%%%%%%%%%%%%%%%%%%%%%%%%%%%%%%%%%%%%%%%%%%%%%%%%%%%%%%%%%%%%% 

\bibliography{mylibrary}
\addcontentsline{toc}{chapter}{Bibliography}

%%%%%%%%%%%%%%%%%%%%%%%%%%%%%%%%%%%%%%%%%%%%%%%%%%%%%%%%%%%%%%%%%%%%%%%%%%%%%%%% 
\end{document}

%%% Local Variables:
%%% mode: latex
%%% TeX-master: t
%%% End:
